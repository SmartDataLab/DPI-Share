\documentclass{ctexart}

% Language setting
% Replace `english' with e.g. `spanish' to change the document language
% \usepackage{authblk}
\usepackage[english]{babel}

% Set page size and margins
% Replace `letterpaper' with`a4paper' for UK/EU standard size
\usepackage[letterpaper,top=2cm,bottom=2cm,left=3cm,right=3cm,marginparwidth=1.75cm]{geometry}
% Useful packages
\usepackage{amsmath}
\usepackage{graphicx}
\usepackage[colorlinks=true, allcolors=blue]{hyperref}

\title{中文小论文模板}
\author{苏锦华  (统计学院) 2021103740}




\begin{document}
\maketitle

\begin{abstract}
      智能合约解决保险逆向选择以及区域异质逆向选择的问题。保险按区域以及按人群划分始终是一个公司风险控制的一环,比如禁止某类人群参保或者提高他们的保费,保司一直在以中心化的媒介对抗风险异质以及克服逆向选择,这个过程不乏投入人力物力,其结果是克服风险异质和逆向选择的劳动的成本最终仍然以费率的调整转嫁到其他参保人群。同时,克服逆向选择、和风险异质所带来的规则限制,其实在收窄参保人群,这样的收窄规则也可能会将一些潜在优质客户排除在外。典型的例子就是灾害保险,因为区域异质性较大,加之单次灾害发生涉及的人群往往是集中的,导致曾经洪水保险的试验失败。
\end{abstract}

\section{智能合约保险的竞争力在哪}

首先智能合约是一种基于区块链的技术,区块链与保险行业有天生的契合性:保险业是区块链应用探索的重要领域,因为保险的“大数法则”与区块链的集体共识具有“基因相似性”,将信任视为核心价值主张的保险行业与天生携带信任基因的区块链技术就是“最佳组合”

去中心化成本

跨区域的信任搭建

扣费公示的溯源,之前的成本结构、赔付案例、现在其实是某种程度上开源、分布式定损

\section{为什么是灾险}

灾险的问题最先需要解决,一是国内相对空白、二是灾险风险异质和逆向选择最为明显。灾险大家相对获取的信息是公开的,非个体性的,大家普遍认可的。对于个性化赔付,引入视频照片留证,对于集体性参保,灾后利用无人机AI定损 都是可以AI可以引入去人工的部分。开源AI天生代表了一种公正性,非人工判定,判断是可重复可泛化可类比

\section{灾险的市场有多大}

美国洪水保险,而国内人口,国内洪水保险。国外地震保险、而中国地震保险。需要一些数据,非公开的就去拿

\section{灾险的市场有多大}



\section{引言}





\bibliographystyle{alpha}
\bibliography{sample}

\end{document}