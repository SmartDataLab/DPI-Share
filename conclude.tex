\documentclass{ctexart}

% Language setting
% Replace `english' with e.g. `spanish' to change the document language
% \usepackage{authblk}
\usepackage[english]{babel}

% Set page size and margins
% Replace `letterpaper' with`a4paper' for UK/EU standard size
\usepackage[letterpaper,top=2cm,bottom=2cm,left=3cm,right=3cm,marginparwidth=1.75cm]{geometry}
% Useful packages
\usepackage{amsmath}
\usepackage{graphicx}
\usepackage[colorlinks=true, allcolors=blue]{hyperref}

\title{去中心化指数保险的总结与思考}
\author{苏锦华}




\begin{document}
\maketitle

\begin{abstract}
      智能合约解决保险逆向选择以及区域异质逆向选择的问题。保险按区域以及按人群划分始终是一个公司风险控制的一环,比如禁止某类人群参保或者提高他们的保费,保司一直在以中心化的媒介对抗风险异质以及克服逆向选择,这个过程不乏投入人力物力,其结果是克服风险异质和逆向选择的劳动的成本最终仍然以费率的调整转嫁到其他参保人群。同时,克服逆向选择、和风险异质所带来的规则限制,其实在收窄参保人群,这样的收窄规则也可能会将一些潜在优质客户排除在外。保险业内耗以及对长尾风险的延迟引发各种信用风险,典型的例子就是灾害保险,因为区域异质性较大,加上长尾风险和区域集中发生的,导致国内灾害保险的失败。智能合约实现的去中心保险,将普通保险公司聚焦于更加纯粹的精算能力上,将长尾风险资金池解耦。这一过程离不开数据劳工和更加优质的承保定损自动化技术,数据和算法更加完善的保险品类,实现去中心化的降本增效潜力也越大。
\end{abstract}

\section{引言}

在对劳动力空间竞争、契约与经济发展等有更深的认识,也同时开始关注跨区域的空间信任问题。银保监会对银行保险业未来三年高质量发展的指导意见也指出,科技促进经济发展的重要地位,总体认可区块链技术对未来银行业和保险业的降本增效作用。当前区块链应用浮于表面,主打溯源可靠,并未充分利用其技术优势解决其中的核心困难。随着技术沉淀,实体经济与数字经济将呈现一体两面的特点。区块链技术的深层次高质量应用是我重点思考的问题,结合我的已有研究经历,我开始重点研究区块链对灾害保险的革新与颠覆。

\section{研究背景}

区块链保险是否是一个伪命题,取决于区块链技术是否能解决保险的核心问题。保险与区块链有天然的基因相似性,通过搭建信任和契约来规避个体性风险事件发生造成的恶劣影响。加之区块链的分布式特点,终态的区块链保险将与现有的保险形式大相径庭,但将更接近精算科学的本质。商业保险本身逐利避险的特点,决定了其对于精算技术的应用天然地希望分辨并摒弃高风险的人群参保,这一特性如同银行贷款如出一辙,随着数据和算法的发展,吸血但不救急的现象只会更加严重。一个明显的证据就是灾害保险最终都被商业保险摒弃,大多数成为政府兜底,对于地方财政的影响是极大的。同时,保司的理赔定损环节也可能滋生店大欺客的现象,随着时间的过渡,不信任危机和道德风险是通常是保司需要重点解决的问题。研究区块链的信任机制对保险规范化作用,以及智能合约对精算纯粹性的帮助。

\subsection{去中心化保险(DI)}

指数保险的智能合约已经有商业模式的成功案例,目前发展最好的是去中心化保险平台是Etherisc。灾险在这个平台有成功的产品,龙卷风保险、作物保险。而研究其成功的模式,并利用精算科学解决这种模式有待完善的问题,有助于解决国内灾险缺失的现状,为地方财政提供风险控制方案,提高社会福利有重大帮助。

指数灾险的智能合约是一个跨科学的应用,涉及AI、区块链以及精算科学,它的核心竞争力在于去中心化的模式解耦了精算科学与风险资金池,使得长尾风险的应用,比如说灾险,变得更加值得信任、也更加精算纯粹。重点技术还是在这个指数保险的智能合约,利用区块链投资的智能合约技术,利用某一个区块链的这个智能合约平台(以太坊、蚂蚁链等)来发布一个去中心化保险(Discentralized Insurance, DI)。



\subsection{自动理赔}


保险有很多种类,去中心化保险最合适的切入点是指数保险,因为指数保险最适合自动化。与遥感相关的那个保险的品类,比如说与农业相关的作物保险、房屋相关的洪水保险、地震保险。与遥感等标准化数据相关的保险,定损过程更加透明,理赔过程自动化是归宿。对于定损自动化的AI算法能力也是重要一环,而不能是把这个功能割裂。技术服务的完整性、整套流程都通过智能合约数据上链的可溯源性使得智能合约保险能够在降本增效上更上一个台阶。指数保险资产合约拥有了可靠的自动定损功能,就相当于一个可以独立运行的保险产品,整套的技术设计可以把把传统保司的运营成本,也就是中心化带来的成本去除。店大欺客所造成的拒绝理赔现象,我们试图打破它。对于传统保司来说,拒绝赔付所获得的就是他的利润,这样的一个事件,其实是造成整个现在我看到保险行业的内耗的一个内生原因,拉客时信誓旦旦、赔付时不思其反,造成的保险乱象导致了投保人的不信任,也制约了整个行业的发展。极度内耗、低效的现象产生,其实大家会对保险公司的信用产生一个质疑,所以就造成了老生常谈的一个现象,我们说中国永远是保险的潜力大国,但是保险覆盖率一直没有推开。所以我们用区块链智能合约这个技术去实现保险模式的重构,去给精算科学带来更加纯粹的研究空间。

\subsection{数据劳工与合约燃气(gas)}

对于上述去中心化保险的盈利模式,其实是将数据、算力、风险资金进行了解耦,收益主要分为固定收益和风险收益。固定收益主要是每次执行合约所涉及的算力消耗与数据购买费用。对于算力消耗,其实是以太坊的基本概念,以太坊作为世界计算机、所谓的矿工提供算力获得的奖励一部分就来源于执行合约所消耗的燃气(gas),也就是说在公链上提供记账的算力是算力劳工,可以获得算力提供的收益。

而获得了牌照的安全数据接口提供商可以通过提供数据的数据体量和质量获得收益,比如航班延误险需要航班数据接口、干旱险需要气象局提供的降水数据。把焦点聚焦于智能合约,现在的智能合约可以接入安全稳固的这个数据接口(比如chainlink就是一家针对智能合约的数据接口提供商)。我们所说的智能合约固定费用,主要就是支付给算力劳工和数据劳工的基本费用,这一部分的费用可以不涉及资本。如果需要的数据是遥感数据,权威的数据接口提供商可能是航天局或者有政府职能的数据产出部门,目前航天局也售出一些遥感产品给现在的商业保险公司,比如说太平洋的农业保险就是根据遥感数据实现了作物产量的监控和部分自动化定损理赔。航天局作为数据劳工出售了原始遥感或者处理后的影像。


\subsection{风险资金池}

上述数据劳工和算力劳工其实都不涉及资本成本,我们说保险的本质是要兜底的,也就是大家会关心最极端的情况发生时自己的权益是否还能得到保障。从精算科学的角度来看,这其实是一个期望风险和长尾风险的问题,保险公司拒绝理赔即获得利润的商业属性导致保险公司会采用一种银行贷款的思路,利用数据把用户划分风险的三流九等后天然的利用条款上的漏洞拒绝赔付偿付长尾风险发生的用户。

但是长尾风险才是购买保险的用户最希望规避的风险,长尾风险的典型就是灾害保险。长尾风险的保险产品是离不开精算科学、大数据和AI的。但从目前国内的现状来看,业界的精算作用甚微,主要作用是面向监管,而学界的精算科学已经是领先业界的状态,国外应用先进精算技术较早,所以在国外无论是龙卷风还是洪水,其实都有成熟稳定的产品,但是在国内由于保险业的内耗或者是模式的劣根性以及一些历史原因灾险市场一直是空白。

Etherisc的设计使得去中心化保险的费用可以解耦为算力和数据劳工的固定费用、期望风险的支出、购买抵御长尾风险波动资本的利息。根据大数定律和精算科学的结论,随着保险的稳定运行,理赔的支出会等于期望风险支出,但不排除黑天鹅或者偶发性事件,那么抵御这部分事件的资金储备则是以类似借款或者风险资产打包出售的模式来实现,为引入这部分备用资金储备所支付的资本利息也是精算科学发力的重点。这种模式成功将风险资金池与商业保险公司的利润解耦,也就是说对这部分风险资金池投资的可以是再保险公司、也就是说智能合约通过制度设计完成了保险公司建立信用的全流程。当然这智能合约技术会完全取代保险公司,对于定损复杂,自动止损技术发展不足的保险品类,中心化仍然占重要作用,普通保险公司也可以完全利用智能合约技术来发布自己的产品,只是这时持有风险资金池的保险公司其实在承担再保险的职能。保险公司主要的职责可以回归精算本身,使得精算科学更加纯粹,使得灾险市场的空白通过解决内耗而重新焕发生机。

\section{讨论与思考}

支付宝的相互宝不等于去中心化保险,支付宝在保单受理和理赔裁决的中心化权力,其规则的频繁改动也是触发了人们的不信任,本质上说定损理赔自动化技术发展不成熟的保险品类,无法实现真正意义上的去中心化保险(DI)。互助保险不等于去中心化保险,核心差异点在于承保定损的自动化技术,所以不能跳脱AI与自动定损算法来谈去中心化保险,所以我们的项目也把AI自动定损放在首位,这也是核心壁垒所在。

其次不能直接跳脱国内商业保险公司来探去中心化保险,本质上保险公司做精算设计,为长尾风险负责。我们的去中心化保险也必须与保司进行合作,保险你没有保险牌照,只能先试验,比如说江苏有一个金融创新监管沙盒,我曾经实习一年多的百姓车联,可以理解为车险领域的相互保,目前他们通过进入这个沙盒,做更多数据驱动的车后维修的优化以及精算深研,去实现比传统保险公司更加全流程化的降本增效,他们目前的产品也最终是通过江苏的紫金财险去做一个合作产品,同时我们不能忽视传统保险公司的销售团队能力。另一方面我们可以看到Etherisc的作物保险,也是提供一个全流程方案给到了一个叫ACRE Africa的一个南非农业保险这个公司。


% \section{导师建议}

% 产品尽快进入市场 -- 陶峰
% 从区块链本质 -- 杨翰方
% 从遥感数据出发 -- 白琰冰




\bibliographystyle{alpha}
\bibliography{sample}

\end{document}